\documentclass[reqno,a4paper,12pt]{amsart}

\usepackage{amsmath,amssymb,amsthm,geometry,xcolor,soul,graphicx}
\usepackage{titlesec}
\usepackage{enumerate}
\usepackage{lipsum}
\usepackage{listings}
\allowdisplaybreaks[4] %align公式跨页
\RequirePackage[most]{tcolorbox}
\usepackage{xeCJK}
\setCJKmainfont{Kai}

\geometry{left=0.7in, right=0.7in, top=1in, bottom=1in}



\renewcommand{\baselinestretch}{1.3}

\title{高等热力学与统计物理第五次作业}
\author{董建宇}

\begin{document}
\maketitle

\titleformat{\section}[hang]{\small}{\thesection}{0.8em}{}{}
\titleformat{\subsection}[hang]{\small}{\thesubsection}{0.8em}{}{}

\section{证明体积为$V$,温度为$T$的辐射场有以下关系:
\begin{align*}
	E &= V\frac{\pi^2(kT)^4}{15(\hbar c)^3} \\
	F &= -\frac{1}{3}E \\
	S &= \frac{4E}{3T} \\
	P &= \frac{1}{3}\frac{E}{V}
\end{align*}}
\begin{tcolorbox}[breakable, colback = black!5!white, colframe = black]
对于辐射场有:化学式$\mu = 0$,能量动量关系为:$\epsilon = c\vert \vec{p} \vert = c\hbar \vert \vec{k} \vert$,服从波色分布,简并度为2。则计算粒子数有:
\[
	N = 2\sum_{\vec{k}} \frac{1}{e^{\beta\epsilon}-1} = 2\int \frac{V}{(2\pi)^3} 4\pi k^2\,dk \frac{1}{e^{\beta\epsilon}-1} = V \int \frac{\epsilon^2}{\pi^2(\hbar c)^3} \frac{1}{e^{\beta\epsilon}-1} \,d\epsilon.
\]
即态密度表达式为: 
\[
	D(\epsilon) = \frac{\epsilon^2}{\pi^2(\hbar c)^3}.
\]
则内能为:
\[
	E = V\int_0^{\infty} D(\epsilon)\frac{\epsilon}{e^{\beta\epsilon}-1}\,d\epsilon = \frac{V}{\pi^2(\hbar c)^3 \beta^4} \int_0^{\infty} \frac{x^3}{e^x-1}\,dx = \frac{V}{\pi^2(\hbar c)^3 \beta^4} \frac{\pi^4}{15} = V\frac{\pi^2(kT)^4}{15(\hbar c)^3}.
\]
其中$\beta = \frac{1}{kT}$,$x = \beta \epsilon$。 \\
巨配分函数为:
\[
	\mathcal{Q} = \prod_{\vec{k}} \frac{1}{1-e^{-\beta\epsilon(\vec{k})}}.
\]
则系统压强为:
\[
	P = \frac{kT}{V}\ln\mathcal{Q} = -\frac{kT}{V}\sum_{\vec{k}} \ln(1-e^{-\beta\epsilon(\vec{k})}) = -\frac{kT}{\pi^2(\hbar c)^3}\int_0^{\infty}\epsilon^2\ln(1-e^{-\beta\epsilon})\,d\epsilon.
\]
计算上述积分:
\begin{align*}
	\mathbf{I} &= \int_0^{\infty} \epsilon^2 \ln(1-e^{-\beta\epsilon})\,d\epsilon = \int_0^{\infty} \ln(1-e^{-\beta\epsilon}) \,d(\epsilon^3/3) \\ 
	&= \left. \frac{1}{3}\epsilon^3 \ln(1-e^{-\beta\epsilon}) \right\vert_0^{\infty} - \frac{\beta}{3}\int_0^{\infty} \frac{\epsilon^3}{e^{\beta\epsilon}-1}\,d\epsilon \\
	&= -\frac{1}{3\beta^3}\int_0^{\infty} \frac{x^3}{e^x-1}\,dx = -\frac{1}{3\beta^3}\frac{\pi^4}{15}.
\end{align*}
则压强为:
\[
	P = \frac{1}{3}\frac{\pi^2(kT)^4}{15(\hbar c)^3} = \frac{1}{3}\frac{E}{V}.
\]
熵为:
\[
	S = \frac{PV+E}{T} = \frac{4}{3}\frac{E}{T}.
\]
Helmholtz自由能为:
\[
	F = E - TS = -\frac{1}{3}E.
\]
\end{tcolorbox}

\section{考虑两维自旋为零的自由boson系统 \\
(1) 推导单位面积的态密度公式; \\
(2) 推导粒子数密度(面密度)用温度和易逸度表达的公式;\\
(3) 证明此系统无凝聚现象}
\begin{tcolorbox}[breakable, colback = black!5!white, colframe = black]
\begin{enumerate}[(1)]
\item 计算粒子数密度N:
\[
	N = \sum_{\vec{k}} \frac{1}{e^{\beta(\epsilon-\mu)}-1} = \int \frac{S}{(2\pi)^2} \,dk_x\,dk_y \frac{1}{e^{\beta(\epsilon-\mu)}-1} = S \int \frac{m}{2\pi\hbar^2} \frac{1}{e^{\beta(\epsilon-\mu)}-1} \,d\epsilon.
\]
则态密度表达式为:
\[
	D(\epsilon) = \frac{m}{2\pi\hbar^2}.
\]

\item 易逸度为:$z = e^{\beta\mu}$,则粒子数密度为:
\[
	\frac{N}{S} = \frac{m}{2\pi\hbar^2} \int_0^{\infty} \frac{ze^{-\beta\epsilon}}{1-ze^{-\beta\epsilon}}\,d\epsilon = \frac{mkT}{2\pi\hbar^2}\int_0^{\infty} \frac{ze^{-x}}{1-ze^{-x}}\,dx = -\frac{mkT}{2\pi\hbar^2} \ln(1-z).
\]

\item 假设该系统有凝聚现象,临界温度为$T_C$,基态能量为$0$,则当系统处在临界温度时,有$\mu \to \epsilon_0- = 0-$即$z \to 1-$:
\[
	N = \lim_{z\to 1-} \frac{Sm}{2\pi\hbar^2}\int_0^{\infty} \frac{ze^{-\epsilon'/(kT_C)}}{1-ze^{-\epsilon'/(kT_C)}}\,d\epsilon' = \lim_{z\to 1-} -\frac{SmkT_C}{2\pi\hbar^2} \ln(1-z) \to +\infty
\]
即假设不成立,所以该系统无凝聚现象。
\end{enumerate}
\end{tcolorbox}

\section{证明在高温或低密度区域$(\rho\lambda^3 << 1)$,自旋为$j$的非相对论自由量子气体的状态方程和熵由下列两式给出:
\[
	PV = NkT\left[1 \pm \frac{\rho\lambda^3}{2^{5/2}(2j+1)} + \cdots \right]
\]
\[
	S = Nk\ln\frac{(2j+1)e^{5/2}}{\rho\lambda^3}\pm Nk\frac{\rho\lambda^3}{2^{7/2}(2j+1)} + \cdots
\]
其中上边的符号对应于fermions,下边的符号对应于bosons,$\lambda$为热波长,$\cdots$代表$\rho\lambda^3$的更高阶项。
}
\begin{tcolorbox}[breakable, colback = black!5!white, colframe = black]
对于自旋为$j$的非相对论自由量子气体,简并度为:$\omega = 2j+1$,则状态方程为:
\[
	\frac{P}{kT} = \frac{2j+1}{\lambda^3}\left( z \mp \frac{z^2}{2^{5/2}} + \cdots \right).
\]
其中易逸度满足:
\[
	\rho = \frac{N}{V} = \frac{2j+1}{\lambda^3}\left( z \mp \frac{z^2}{2^{3/2}} + \cdots \right).
\]
级数反演得:
\[
	z = \frac{\rho\lambda^3}{2j+1} \pm \frac{1}{2^{3/2}} \left( \frac{\rho\lambda^3}{2j+1} \right)^2 + \cdots.
\]
则状态方程可以写为:
\[
	\frac{P}{kT} = \frac{N}{V} \left( 1 \pm \frac{1}{2^{5/2}}\frac{\rho\lambda^3}{2j+1} + \cdots \right)
\]
理想费米(波色)气体的巨配分函数的对数为:
\[
	\ln \Xi = \pm \sum_l \omega_l \ln\left( 1 \pm e^{-\alpha-\beta\varepsilon_l} \right)
\]
即:
\[
	\ln \Xi = \pm g \frac{2\pi V (2m)^{3/2}}{h^3} \int \sqrt{\varepsilon} \ln\left( 1 \pm e^{-\alpha-\beta\varepsilon} \right)\,d\varepsilon .
\]
在弱兼并条件下,理想费米(波色)气体的压强为:
\begin{align*}
	p &= \frac{1}{\beta}\frac{\partial}{\partial V}\ln \Xi \\
	&= \pm \frac{2\pi g(2m)^{3/2}}{h^3\beta} \int \sqrt{\varepsilon} \left( \pm e^{-\alpha-\beta\varepsilon} - \frac{1}{2}e^{-2\alpha-2\beta\varepsilon} \right) \,d\varepsilon \\
	&= \frac{g(2\pi m)^{3/2}}{h^3\beta^{5/2}} e^{-\alpha} \left( 1 \mp \frac{1}{4\sqrt{2}}e^{-\alpha} \right) \\
	&= \frac{NkT}{V} \left( 1 \pm \frac{1}{4\sqrt{2}}e^{-\alpha} \right)
\end{align*}
其中,利用玻尔兹曼分布近似有:
\[
	e^{-\alpha} = \frac{N}{V} \frac{h^3}{(2\pi m kT)^{3/2}} \frac{1}{g}.
\]
则理想费米(波色)气体的压强为:
\[
	p = \frac{NkT}{V} \left( 1 \pm \frac{1}{4\sqrt{2}}\frac{N}{V} \frac{h^3}{(2\pi m kT)^{3/2}} \frac{1}{g} \right).
\]
弱简并理想费米(波色)气体的内能为:
\[
	U = \frac{3}{2}NkT\left( 1 \pm \frac{1}{4\sqrt{2}}\frac{N}{V} \frac{h^3}{(2\pi m kT)^{3/2}} \frac{1}{g} \right).
\]
等容热熔为:
\[
	C_V = \left( \frac{\partial U}{\partial T} \right)_V = \frac{3}{2}Nk \left( 1 \mp \frac{1}{8\sqrt{2}}\frac{N}{V}\frac{h^3}{(2\pi mkT)^{3/2}} \frac{1}{g} \right)
\]
则弱简并理想费米(波色)气体的熵为:
\begin{align*}
	S &= \int \frac{C_V}{T}\,dT + S_0(V) \\
	&= \frac{3}{2}Nk\left( \ln T \pm \frac{1}{12\sqrt{2}} \frac{N}{V} \frac{h^3}{(2\pi mkT)^{3/2}} \frac{1}{g} \right) + S_0(V).
\end{align*}
当$\frac{N}{V}\left( \frac{h}{\sqrt{2\pi mkT}} \right)^3 << 1$时,弱简并理想费米(波色)气体趋于经典理想气体,则有:
\[
	\frac{3}{2}Nk\ln T + S_0(V) = \frac{3}{2}Nk\ln T + Nk\ln \frac{gV}{N} + \frac{3}{2}Nk \left[ \frac{5}{3} + \ln\left( \frac{2\pi mk}{h^2} \right) \right]
\]
则弱简并理想费米(波色)气体的熵为:
\[
	S = Nk\left( \frac{5}{2} \pm \frac{1}{8\sqrt{2}}\frac{N}{V}\frac{h^3}{(2\pi mkT)^{3/2}}\frac{1}{g} + \frac{3}{2}\ln \frac{2\pi mkT}{h^2} + \ln \frac{gV}{N} \right).
\]
其中$g = \omega = 2j+1$,则有:
\[
	S = Nk\ln \frac{(2j+1)e^{5/2}}{\rho\lambda^3} \pm Nk \frac{\rho\lambda^3}{2^{7/2}(2j+1)} + \cdots.
\]
\end{tcolorbox}
\end{document}