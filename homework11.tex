\documentclass[reqno,a4paper,12pt]{amsart}

\usepackage{amsmath,amssymb,amsthm,geometry,xcolor,soul,graphicx}
\usepackage{titlesec}
%\usepackage{amssymb}
\usepackage{enumerate}
\usepackage{lipsum}
\usepackage{listings}
\allowdisplaybreaks[4] %align公式跨页
\RequirePackage[most]{tcolorbox}
\usepackage{xeCJK}
\setCJKmainfont{Kai}

\geometry{left=0.6in, right=0.5in, top=1in, bottom=1in}



\renewcommand{\baselinestretch}{1.3}

\title{高等热力学与统计物理第十一次作业}
\author{董建宇}

\begin{document}
\maketitle

\begin{enumerate}[1.]

\item 考虑近邻作用的铁磁性Ising模型,写出自由能
\[
	F_I(M) \equiv -\frac{kT}{\mathcal{N}} \ln Q_{N_\uparrow}.
\]
在零磁场下展开到$M^4$的形式,讨论此展开式在$T>T_c$和$T<T_c$的图像并求出两种情况下$F_I(M)$的最小值和相应的磁化强度$M$。(此类展开只适用于临界温度附近,故展开式中各项的系数只需要保留到$\vert T-T_c \vert$的领头阶。)
\begin{tcolorbox}[breakable, colframe = black, colback = black!5!white]
对于近邻作用的Ising模型,有:
\[
	Q_{N_\uparrow} = e^{-\frac{U_I}{kT}} \frac{\mathcal{N}!}{N_\uparrow!N_\downarrow!}.
\]
当$\mathcal{N},N_\uparrow,N_\downarrow$充分大时,可以利用Stirling公式得:
\[
	\frac{1}{\mathcal{N}}\ln Q_{N_\uparrow} = -\frac{U_I}{\mathcal{N}kT} - \frac{N_\uparrow}{\mathcal{N}} \ln\frac{N_\uparrow}{\mathcal{N}} - \frac{N_\downarrow}{\mathcal{N}} \ln\frac{N_\downarrow}{\mathcal{N}}.
\]
磁化强度为:
\[
	M = \frac{1}{\mathcal{N}}(N_\uparrow-N_\downarrow)
\]
可得:
\[
	\frac{1}{\mathcal{N}}\ln Q_{N_\uparrow} = \frac{\mu H}{kT}M -\frac{n\varepsilon M^2}{2kT} - \frac{1+M}{2}\ln\frac{1+M}{2} - \frac{1-M}{2}\ln\frac{1-M}{2}.
\]
则自由能为:
\[
	F_I(M) = -\frac{kT}{\mathcal{N}}\ln Q_{N_\uparrow} = -\mu HM + \frac{1}{2}n\varepsilon M^2 + \frac{kT(1+M)}{2}\ln\frac{1+M}{2} + \frac{kT(1-M)}{2}\ln\frac{1-M}{2}
\]
对于铁磁材料,$\varepsilon<0$,则临界温度为:$T_c = -n\varepsilon/k$。在零磁场下,展开式系数只保留到$\vert T - T_c \vert$的领头阶,则有:
\[
	F_I(M) = -kT\ln2 + \frac{1}{2}k(T-T_c) M^2 + \frac{1}{12}kT M^4.
\]
对$M$求导得:
\[
	\frac{\partial F_I(M)}{\partial M} = k(T-T_c)M + \frac{1}{3}kTM^3.
\]
当$T>T_c$时,有$\frac{\partial F_I(M)}{\partial M} \geq 0$,即当$M = 0$时$F_I(M)$取最小值为$-kT\ln2$。 \\
当$T<T_c$时,一阶导数为0可得:
\[
	M = 0,~\text{or}~\pm\sqrt{\frac{3(T_c-T)}{T}}.
\]
此时$F_I(M)$最小值为:$-kT\ln2-\cfrac{3}{4}\cfrac{k(T_c-T)^2}{T}$,此时$M = \pm \sqrt{\frac{3(T_c-T)}{T}}$。

%当$T>T_c$时,一阶导数恒大于0,即不存在$F_I(M)$的最小值。当$T<T_c$时,使$F_I(M)$取最小值时$M = \sqrt{\frac{3(T_c-T)}{T}}$
\end{tcolorbox}


\item 求出具有周期边界条件和仅有近邻相互作用$u$的一维格气巨配分函数$\mathcal{Q}(y) = 0$的根$y_1,y_2,\dots,y_{\mathcal{N}}$,并讨论在$u>0$和$u<0$两种情况下根在复平面上的分布。
\begin{tcolorbox}[breakable, colframe = black, colback = black!5!white]
具有周期边界条件同时仅有紧邻相互作用的一维Ising模型的配分函数为:
\[
	Q_I = \lambda_+^{\mathcal{N}} + \lambda_-^{\mathcal{N}}.
\]
其中
\begin{align*}
	\lambda_\pm =& \frac{1}{2}\left[ x_I\left( y_I+\frac{1}{y_I} \right) \pm \sqrt{x_I^2\left( y_I - \frac{1}{y_I} \right)^2 + \frac{4}{x_I^2}} \right], \\
	x_I =& e^{-\frac{1}{kT}\varepsilon}, \\
	y_I =& e^{\frac{\mu H}{kT}}.
\end{align*}
对应一维格气巨配分函数可表示为:
\[
	\mathcal{Q} = Q_I e^{\frac{\mathcal{N}(\mu H+\varepsilon)}{kT}} = (\lambda_+^\mathcal{N} + \lambda_-^\mathcal{N}) e^{\frac{\mathcal{N}(\mu H+\varepsilon)}{kT}} = e^{\frac{\mathcal{N}\varepsilon}{kT}} y_I^\mathcal{N}(\lambda_+^\mathcal{N} + \lambda_-^\mathcal{N}).
\]
令巨配分函数为0,则有:
\[
	\lambda_+ = \lambda_- \exp\left( i(2k+1)\pi/N \right)
\]
其中$k \in N,~y_I^2 = ye^{-u/kT}$,则$y$满足的方程为:
\[
	y^2 + 2y\exp\left( \frac{u}{kT} \right)\left[ 1+\exp\left( \frac{u}{kT} \right) \cos^2\left( \frac{(2k+1)\pi}{2N} \right) \left( \exp(u/kT)-1 \right) \right] + \exp\left( \frac{2u}{kT} \right) = 0.
\]
当$u<0$时,有:
\begin{align*}
	y =& -\exp\left( \frac{u}{kT} \right)\left[ 1 + 4\exp\left( \frac{u}{kT} \right) \cos^2\frac{(2k+1)\pi}{2N} \left( \exp\left( \frac{u}{kT} \right) - 1 \right) \right] \\
	&\pm i \exp\left( \frac{u}{kT} \right) \sqrt{1 - \left[ 1+4\exp\left( \frac{u}{kT} \right) \cos^2\frac{(2k+1)\pi}{2N} \left( \exp\left( \frac{u}{kT} \right) - 1 \right) \right]^2}
\end{align*}
此时,所有根分布在半径为$\exp(u/kT)$的圆上。 \\
当$u>0$时,有:
\begin{align*}
	y =& -\exp\left( \frac{u}{kT} \right)\left[ 1 + 4\exp\left( \frac{u}{kT} \right) \cos^2\frac{(2k+1)\pi}{2N} \left( \exp\left( \frac{u}{kT} \right) - 1 \right) \right] \\
	&\pm \exp\left( \frac{u}{kT} \right) \sqrt{\left[ 1+4\exp\left( \frac{u}{kT} \right) \cos^2\frac{(2k+1)\pi}{2N} \left( \exp\left( \frac{u}{kT} \right) - 1 \right) \right]^2 - 1}
\end{align*}
%因为有:
%\[
%	y_I\lambda_\pm = \frac{1}{2}\left[ x_I(y_I^2+1) \pm \sqrt{x_I^2 \left( y_I^2-1 \right)^2 + \frac{4y_I^2}{x_I^2}} \right]
%\]
%则巨配分函数可以写为:
%\[
%	\mathcal{Q} = e^{\frac{\mathcal{N}\varepsilon}{kT}} \sum_{n=0}^{\mathcal{N}} \binom{\mathcal{N}}{n} 
%\]
\end{tcolorbox}
%\lfloor \mathcal{N}/2 \rfloor

\item 当$\mathcal{N}\to\infty$时,证明上一题讨论的一维格气的压强为
\[
	\frac{P}{kT} = \ln \left[ 1+\frac{y}{x} + \sqrt{\left( 1-\frac{y}{x} \right)^2 + 4y} \right] - \ln2
\]
并求出粒子的数密度。其中
\[
	x = e^{\frac{u}{kT}}
\]
\begin{tcolorbox}[breakable, colframe = black, colback = black!5!white]
当$\mathcal{N}\to \infty$时,配分函数为:
\[
	Q_I = \lambda_+^\mathcal{N} + \lambda_-^\mathcal{N} \approx \lambda_+^\mathcal{N}
\]
则巨配分函数为:
\[
	\mathcal{Q} = e^{\frac{\mathcal{N}\varepsilon}{kT}} y_I^{\mathcal{N}} \lambda_+^\mathcal{N} = \frac{1}{2}\left[ 1+\frac{y}{x} + \sqrt{\left( 1-\frac{y}{x} \right)^2 + 4y} \right].
\]
则压强为:
\[
	\frac{P}{kT} = \ln \mathcal{Q} =  \ln \left[ 1+\frac{y}{x} + \sqrt{\left( 1-\frac{y}{x} \right)^2 + 4y} \right] - \ln2.
\]
粒子数密度为:
\[
	n = y\frac{\partial }{\partial y}\ln \mathcal{Q} = y\frac{\frac{1}{x} + \frac{-\frac{1}{x}(1-\frac{y}{x}) + 4}{\sqrt{( 1-\frac{y}{x})^2 + 4y}}}{1+\frac{y}{x} + \sqrt{\left( 1-\frac{y}{x} \right)^2 + 4y}}
\]
\end{tcolorbox}

\end{enumerate}

\end{document}