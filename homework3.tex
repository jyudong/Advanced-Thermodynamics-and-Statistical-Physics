\documentclass[reqno,a4paper,12pt]{amsart}

\usepackage{amsmath,amssymb,amsthm,geometry,xcolor,soul,graphicx}
\usepackage{titlesec}
\usepackage{enumerate}
\usepackage{lipsum}
\usepackage{listings}
\RequirePackage[most]{tcolorbox}
\usepackage{xeCJK}
\setCJKmainfont{Kai}

\geometry{left=0.7in, right=0.7in, top=1in, bottom=1in}



\renewcommand{\baselinestretch}{1.3}

\title{高等热力学与统计物理第三次作业}
\author{董建宇}


\begin{document}
\maketitle

\titleformat{\section}[hang]{\small}{\thesection}{0.8em}{}{}
\titleformat{\subsection}[hang]{\small}{\thesubsection}{0.8em}{}{}

\begin{enumerate}[1]
	\item 证明大整数N阶乘的Stirling公式
	\[
		\ln N! = N(\ln N - 1) + \mathcal{O}(\ln N)
	\]
	其中$\mathcal{O}(\ln N)$表示误差与$\ln N$同阶。(一个简单的证明方法是考虑用矩形法近似积分$\int_1^{N} \,dx\ln x.$)
	\begin{tcolorbox}[breakable, colback = black!5!white, colframe = black]
	使用分部积分计算可知:
	\[
		\int_1^N \,dx \ln x = x\ln x \vert_1^{N} - \int_1^N \frac{1}{x} x\,dx = N\ln N - N.
	\]
	将积分区域划分为$(N-1)$份长度为$1$的区间,则积分近似为:
	\[
		\int_1^N \,dx \ln x \approx \sum_{i=1}^{N} \ln i = \ln N!.
	\]
	要证明误差与$\ln N$同阶,即证明存在$x \in R$,当$N$充分大时满足
	\[
		\ln N! - N\ln N + N = x\ln N.
	\]
	令$g(N) = \ln N! - N\ln N + N$,则有:
	\[
		g(N+1) - g(N) = 1 - N\ln (1+1/N) > 0.
	\]
	即$g(N)$为单调递增数列,$g(N) \geq g(1) = 1$. \\
	令$f_{N}(x) = \ln N! - (N+x)\ln N + N$,则有:
	\[
		\frac{df_{N}(x)}{dx} = -\ln N < 0
	\]
	则$f_{N}(x)$为单调递减函数,$f_{N}(0) = g(N) > 0$,当$x = 2$时,
	\[
		f_N(2) = \ln N! - (N+2)\ln N + N.
	\]
	计算差分
	\[
		f_{N+1}(2) - f_N(2) = 1 - \ln\left( 1+\frac{1}{N} \right)^N -2\ln\left( 1+\frac{1}{N} \right) < 0.
	\]
	则数列$f_N(2)$为单调递减数列。
	\[
		f_3(2) = \ln 3! - 5\ln 3 + 3 \approx -0.7 < 0.
	\]
	即证明了对于充分大的$N$,存在$x_0 \in (0,2)$,满足
	\[
		f_N(x_0) = 0.
	\]
	即有
	\[
		\ln N! = N(\ln N - 1) + \mathcal{O}(\ln N).
	\]
	\end{tcolorbox}

	\item 考虑由$M$个相同的$S$系统,$M'$个相同的$S'$系统等等组成的一个正则系宗。系宗中的系统处在不同的位置但相互热接触。令系统$S$, $S'$,$\cdots$的Hamiltonian为$H$, $H'$, $\cdots$,其本征态和本征能量由下列方程给出
	\begin{align*}
		&H\psi_j = E_j\psi_j \\
		&H'\psi_j' = E_j'\psi_j' \\
		&\cdots \cdots
	\end{align*}
	证明:找到某一特定系统$S$处于$\psi_j$的状态的几率是
	\[
		P_j = \frac{1}{Q}e^{-\beta E_j}
	\]
	找到某一特定系统$S$处于$\psi_j$的状态的几率是
	\[
		P_j' = \frac{1}{Q'}e^{-\beta E_j'}
	\]
	其中$Q = \sum_j e^{-\beta E_j}$, $Q' = \sum_j e^{-\beta E_j'}$.
	\begin{tcolorbox}[breakable, colback = black!5!white, colframe = black]
	系统总Hamiltonian为:
	\[
		H_0 = H + H'.
	\]
	令$M_j$为处于$H$的本征态对应能量$E_j$的系统数目,$M_j'$为处于$H'$的本征态对应能量$E_j'$的系统数目,则有:
	\begin{align*}
		&\sum_j M_j = M, \\
		&\sum_j M_j' = M', \\
		&\sum_j M_jE_j + \sum_j M_j'E_j' = \mathcal{E}.
	\end{align*}
	系统对应$\{ M_j, M_j' \}$的状态数为:
	\[
		\Omega\{ M_j, M_j' \} = \frac{M!}{\prod_j M_j!}\frac{M'!}{\prod_jM_j'!}.
	\]
	则有:
	\[
		\ln \Omega = \ln M! - \sum_j \ln M_j! + \ln M'! - \sum_j \ln M_j'!.
	\]
	引入不定乘子$\alpha$,$\alpha'$,$\beta$,则有:
	\begin{align*}
		&\frac{\partial}{\partial M_j} \left( \ln \Omega + \alpha M + \alpha' M' + \beta \mathcal{E} \right) = 0; \\ 
		&\frac{\partial}{\partial M_j'} \left( \ln \Omega + \alpha M + \alpha' M' + \beta \mathcal{E} \right) = 0.
	\end{align*}
	利用Stirling公式可得:
	\[
		M_j = e^{-\alpha-\beta E_j}, ~ M_j' = e^{-\alpha'-\beta E_j'}.
	\]
	则找到某一特定系统$S$处于$\psi_j$状态的几率为:
	\[
		P_j = \frac{M_j}{\sum_j M_j} = \frac{e^{-\beta E_j}}{Q}.
	\]
	找到某一特定系统$S'$处于$\psi'$状态的几率是
	\[
		P_j' = \frac{M_j'}{\sum_j M_j'} = \frac{e^{-\beta E_j'}}{Q'}.
	\]
	\end{tcolorbox}
	
	\item 证明巨正则系综的最可几分布内粒子数涨落为
	\[
		\frac{\Delta N}{\langle N \rangle} = \sqrt{\frac{kT\rho \kappa_T}{\langle N \rangle}}
	\]
	其中$\rho = \frac{\langle N \rangle}{V}$为密度, $\Delta N^2 = (N - \langle N \rangle)^2$的平均值(均方偏差),$\kappa_T$为等温压缩系数。由此可见$\kappa_T > 0$。
	\begin{tcolorbox}[breakable, colback = black!5!white, colframe = black]
	在巨正则系统中,系统具有$N$个粒子且处于$H(N)$的本征态$\psi_{j(N)}$的几率为:
	\[
		P_{j(N)} = \frac{1}{Q}e^{-\beta E_{j(N)} - \gamma N}.
	\]
	则可以计算:
	\[
		\langle N \rangle = \frac{\sum_N \sum_{j(N)} Ne^{-\beta E_{j(N)} - \gamma N}}{Q}.
	\]
	\begin{align*}
		-\frac{\partial \langle N \rangle}{\partial \gamma} &= -\frac{\partial }{\partial \gamma}\left( \frac{\sum_{N}\sum_{j(N)} Ne^{-\beta E_{j(N)} - \gamma N}}{Q} \right) \\
		&= \frac{Q\sum_N \sum_{j(N)} N^2e^{-\beta E_{j(N)} - \gamma N} - \left( \sum_N\sum_{j(N)} N e^{-\beta E_{j(N)} - \gamma N} \right)^2}{Q^2} \\
		&= \langle N^2 \rangle - \langle N \rangle^2 \\
		&= \Delta N^2.
	\end{align*}
	利用$\gamma = -\frac{\mu}{kT}$,有:
	\[
		\Delta N^2 = kT \left( \frac{\partial \langle N \rangle}{\partial \mu} \right)_{T,V}.
	\]
	由吉布斯自由能全微分可知:
	\[
		dG = -S\,dT + V\,dP + \mu\,d\langle N \rangle = \mu\,d\langle N \rangle + \langle N \rangle \,d\mu.
	\]
	则有:
	\[
		\,d\mu = \frac{V}{\langle N \rangle}\,dP - \frac{S}{\langle N \rangle}\,dT = v\,dp - s\,dT.
	\]
	其中$\mu$,$v$和$s$为单个粒子的化学式,体积和熵。则有:
	\[
		\left( \frac{\partial \mu}{\partial v} \right)_T = v \left( \frac{\partial p}{\partial v} \right)_T
	\]
	注意到$v = \frac{V}{\langle N \rangle}$,在体积不变而粒子数变化的情况下有:
	\[
		v\left( \frac{\partial p}{\partial v} \right)_T = \left( \frac{\partial \mu}{\partial \langle N \rangle} \right)_T \left( \frac{\partial \langle N \rangle}{\partial v} \right)_T = -\frac{\langle N \rangle^2}{V} \left( \frac{\partial \mu}{\partial N} \right)_{T,V}.
	\]
	则有:
	\[
		\frac{\Delta N}{\langle N \rangle} = \sqrt{\frac{kT}{\langle N \rangle^2} \left( \frac{\partial \langle N \rangle}{\partial \mu} \right)_{T,V}} = \sqrt{-\frac{kT}{V}\frac{1}{v}\left( \frac{\partial v}{\partial p} \right)_T} = \sqrt{\frac{kT\rho\kappa_T}{\langle N \rangle}}.
	\]
	\end{tcolorbox}
\end{enumerate}
\end{document}