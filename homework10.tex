\documentclass[reqno,a4paper,12pt]{amsart}

\usepackage{amsmath,amssymb,amsthm,geometry,xcolor,soul,graphicx}
\usepackage{titlesec}
\usepackage{enumerate}
\usepackage{lipsum}
\usepackage{listings}
\allowdisplaybreaks[4] %align公式跨页
\RequirePackage[most]{tcolorbox}
\usepackage{xeCJK}
\setCJKmainfont{Kai}

\geometry{left=0.4in, right=0.35in, top=1in, bottom=1in}



\renewcommand{\baselinestretch}{1.3}

\title{高等热力学与统计物理第十次作业}
\author{董建宇}

\begin{document}
\maketitle

\begin{enumerate}[1.]

\item 令$\mathcal{Q}_G$是具有两体势能为:
\[
	u(r) = \left\{ \begin{array}{ll}
		\infty, & \text{如果$r=0$}; \\
		u_l, & \text{如果$r=$第$l$近邻。}
	\end{array} \right.
\]
的格气的巨配分函数。 \\
证明:只要令
\[
	N = N_{\uparrow}
\]
\[
	y = \exp\left\{ \frac{1}{kT} \left( 2\mu H - \sum_l n_l [\varepsilon_l(\uparrow \downarrow) - \varepsilon_l(\uparrow\uparrow)] \right) \right\}
\]
\[
	u_l = 2[\varepsilon_l(\uparrow) - \varepsilon_l(\uparrow\downarrow)]
\]
则对于Ising模型配分函数$Q_I$格气配分函数可以表示为
\[
	\mathcal{Q}_G = Q_I \exp\left\{ \frac{\mathcal{N}}{kT} \left[ \mu H + \frac{1}{2}\sum_l n_l\varepsilon_l(\uparrow\uparrow) \right] \right\}
\]
其中$\mathcal{N}$是格点总数,$y$是格气的易逸度,$\varepsilon_l(\uparrow\uparrow) = \varepsilon(\downarrow\downarrow)$表示Ising模型中第l近邻自旋平行的相互作用能,$\varepsilon_l(\uparrow\downarrow)$是相应的自旋反平行的作用能,$n_l$为每一格点第$l$近邻的数目。
\begin{tcolorbox}[breakable, colback = black!5!white, colframe = black]
Ising模型能量为:
\[
	U_I = \sum_l [N_{\uparrow\uparrow}^l \varepsilon_l(\uparrow\uparrow) + N_{\downarrow\downarrow}^l \varepsilon_l(\downarrow\downarrow) + N_{\uparrow\downarrow}^l \varepsilon_l(\uparrow\downarrow)] + \mu H (N_\uparrow - N_\downarrow).
\]
设每个格点第$l$近邻的数目为$n_l$,则有:
\begin{align*}
	2N_{\uparrow\uparrow}^l + N_{\uparrow\downarrow} = n_l N_\uparrow; \\
	2N_{\downarrow\downarrow}^l + N_{\uparrow\downarrow} = n_l N_\downarrow.
\end{align*}
同时,有隔点数目为$\mathcal{N}$:
\[
	N_\uparrow + N_\downarrow = \mathcal{N}
\]
则可以用$N_\uparrow$和$N_{\uparrow\uparrow}^l$表示Ising模型能量:
\small
\[
	U_I = 2\sum_l N_{\uparrow\uparrow}^l [\varepsilon^l(\uparrow\uparrow) - \varepsilon^l(\uparrow\downarrow)] + N_\uparrow^l \left\{ \sum_l n_l [\varepsilon^l(\uparrow\uparrow) - \varepsilon^l(\uparrow\downarrow)] - 2\mu H \right\} + \mathcal{N} \left[ \frac{1}{2}\sum_l n_l \varepsilon_l(\uparrow\uparrow) + \mu H \right]
\]
\normalsize
则Ising模型的巨配分函数为
\[
	Q_I = \sum e^{-U_I/kT}.
\]
其中求和是对所有自旋分布求和。 \\
格气的巨配分函数可表为
\[
	Q_G = \sum_N y^N \frac{1}{N!} \sum_{\text{N个可区分粒子的分布}} e^{-U_G/kT} = \sum_N y^N \sum_{\text{N个不可区分粒子的分布}} e^{-\sum_l n_{pp}^lu_l /kT}.
\]
令$N = N_\uparrow$, $n_{pp}^l = N_{\uparrow\uparrow}^l$, $y = \exp\{ (2\mu H - \sum_l n_l[\varepsilon_l(\uparrow\downarrow - \varepsilon_l(\uparrow\uparrow))])/kT \}$, $u_l = 2[\varepsilon_l(\uparrow\uparrow) - \varepsilon_l(\uparrow\downarrow)]$
则可以计算:
\small
\begin{align*}
	Q_G =& \sum_{N_\uparrow = 1}^\mathcal{N} \exp \left\{ \frac{N_\uparrow}{kT}(2\mu H - \sum_l n_l[\varepsilon_l(\uparrow\downarrow) - \varepsilon_l(\uparrow\uparrow)]) \right\} \sum_{\text{$N_\uparrow$个$\uparrow$的分布}} \exp\left\{ -\frac{1}{kT} \sum_l N_{\uparrow\uparrow}^l 2[\varepsilon_l(\uparrow\uparrow) - \varepsilon_l(\uparrow\downarrow)] \right\} \\
	=& \sum_{\text{$\uparrow$的分布}}\exp\left\{ -\frac{1}{kT} \left[ \sum_l N_{\uparrow\uparrow}^l 2[\varepsilon_l(\uparrow\uparrow)-\varepsilon_l(\uparrow\downarrow)] - N_\uparrow [(\varepsilon_l(\uparrow\uparrow)-\varepsilon_l(\uparrow\downarrow))] + 2\mu H \right] \right\} \\
	=& \sum \exp \left\{ \frac{\mathcal{N}}{kT} \left[ \mu H + \frac{1}{2} \sum_l n_l \varepsilon_l(\uparrow\uparrow) \right] - \frac{1}{kT} U_I \right\} \\
	=& Q_I \exp \left\{ \frac{\mathcal{N}}{kT} \left[ \mu H + \frac{1}{2}\sum_l n_l \varepsilon_l(\uparrow\uparrow) \right] \right\}
\end{align*}


\end{tcolorbox}

\end{enumerate}

\end{document}