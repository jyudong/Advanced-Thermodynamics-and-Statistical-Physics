\documentclass[reqno,a4paper,12pt]{amsart}

\usepackage{amsmath,amssymb,amsthm,geometry,xcolor,soul,graphicx}
\usepackage{titlesec}
\usepackage{enumerate}
\usepackage{lipsum}
\usepackage{listings}
\allowdisplaybreaks[4] %align公式跨页
\RequirePackage[most]{tcolorbox}
\usepackage{xeCJK}
\setCJKmainfont{Kai}

\geometry{left=0.7in, right=0.7in, top=1in, bottom=1in}



\renewcommand{\baselinestretch}{1.3}

\title{高等热力学与统计物理第九次作业}
\author{董建宇}

\begin{document}
\maketitle

\begin{enumerate}[1.]

\item 一稀薄气体处于外力场内,相应的势能为$V(\vec{r})$。假设$V(\vec{r})$在分子相互作用力程范围内的变化很小,求出Boltzmann方程的近似静态解并用平均数密度平均动能定义的温度表示所得得解。
\begin{tcolorbox}[breakable, colback = black!5!white, colframe = black]
处于势场$V(\vec{r})$中受力为:
\[
	\vec{F}(r) = -\vec{\nabla} V(\vec{r}).
\]
则Boltzmann方程可以写为:
\begin{align*}
	&\frac{\partial }{\partial t} f(\vec{r},\vec{p};t) + \left( \frac{\vec{p}}{m}\cdot\vec{\nabla}_{\vec{r}} + \vec{F}\cdot\vec{\nabla}_{\vec{p}} \right) f(\vec{r},\vec{p};t) \\
	=& \int \,d^3\vec{p}_2 \,d^3\vec{p}_1' \,d^3\vec{p}_2' \vert T_{fi} \vert^2 \delta^4(P_f-P_i) [f(\vec{r},\vec{p}_1';t)f(\vec{r},\vec{p}_2';t) - f(\vec{r},\vec{p};t)f(\vec{r},\vec{p}_2;t)].
\end{align*}
假设Boltzmann方程的试探解为:
\[
	f(\vec{r},\vec{p};t) = C\rho e^{-A(\vec{p}-\vec{p}_0)^2}
\]
则坐标空间中粒子数密度为:
\[
	\rho = \int \,d^3\vec{p} f(\vec{r},\vec{p};t) = C \rho \int \,d^3\vec{p} e^{-A(\vec{p}-\vec{p}_0)^2} = C \rho \left( \frac{\pi}{A} \right)^{3/2}.
\]
则有系数间关系为:
\[
	C = \left( \frac{A}{\pi} \right)^{3/2}
\]
则考虑动量空间中粒子数密度为:$g(\vec{p};t) = \frac{f(\vec{r},\vec{p};t)}{\rho(\vec{r};t)} = \left( \frac{A}{\pi} \right)^{3/2} e^{-A(\vec{p}-\vec{p}_0)^2}$。则平均动量为:
\[
	\langle \vec{p} \rangle = \int \,d^3\vec{p} \vec{p} \left( \frac{A}{\pi} \right)^{3/2} e^{-A(\vec{p}-\vec{p}_0)^2} = \left( \frac{A}{\pi} \right)^{3/2} \int \,d^3\vec{p} (\vec{p}+\vec{p}_0) e^{-A\vec{p}\.^2} = \vec{p}_0.
\]
平均动量为0条件即为$\vec{p}_0 = 0$。此时平均动能为:
\[
	\epsilon \equiv \langle \frac{\vec{p}^2}{2m} \rangle = \left( \frac{A}{\pi} \right)^{3/2} \frac{1}{2m} \int \,d^3\vec{p}\, \vec{p}\,^2 e^{-A\vec{p}\.^2} = \frac{3}{4mA}.
\]
若定义平均动能对应的温度为$T_0 = \frac{2\epsilon}{3k}$,其中$k$为Boltzmann常数,则系数可以写为:
\[
	A = \frac{1}{2mkT_0}.
\]
对于该试探解,Boltzmann方程右侧为0,则有:
\[
	\frac{\partial}{\partial t} \rho(\vec{r};t) + \frac{\vec{p}}{m} \cdot \vec{\nabla}_{\vec{r}} \rho(\vec{r};t) - \vec{\nabla}_{\vec{r}} V(\vec{r}) \cdot (-2A\vec{p})\rho(\vec{r};t) = 0.
\]
通解为:
\[
	\rho(\vec{r}) = C_1 e^{-2mAV(\vec{r})}.
\]
其中$C_1$满足:
\[
	\int \,d^3\vec{r} \rho(\vec{r}) = N.
\]
$N$为气体分子总数。则Boltzmann方程近似静态解为:
\[
	f(\vec{r},\vec{p}) = \frac{C_1}{(2\pi mkT_0)^{3/2}} \exp\left( -\frac{V(\vec{r})}{kT_0}-\frac{\vec{p}^2}{2mkT_0} \right).
\]
\end{tcolorbox}

\item 写下一个均匀且无外力作用的气体的Boltzmann方程并证明下列Boltzmann H-定理:
\[
	\frac{dH}{dt} \leq 0
\]
其中$H \equiv \int \,d^3\vec{p} f(\vec{p},t)\ln f(\vec{p},t)$。
\begin{tcolorbox}[breakable, colback = black!5!white, colframe = black]
对于均匀且无外力作用的气体,有:
\[
	\vec{\nabla}_{\vec{r}}f = 0, ~ \vec{F} = 0.
\]
则Boltzmann方程可以写为:
\[
	\frac{\partial}{\partial t} f(\vec{p};t) = \int \,d^3\vec{p}_2\,d^3\vec{p}_1'\,d^3\vec{p}_2' \vert T_{fi} \vert^2 \delta^4(P_f-P_i) [f(\vec{p}_1';t)f(\vec{p}_2';t) - f(\vec{p};t)f(\vec{p}_2;t)].
\]
可以计算:
\begin{align*}
	\frac{dH}{dt} =& \int \,d^3\vec{p} \frac{\partial}{\partial t} ( f(\vec{p},t)\ln f(\vec{p},t) ) \\
	=& \int \,d^3\vec{p} \frac{\partial}{\partial t} f(\vec{p};t) + \int \,d^3\vec{p} \ln f(\vec{p};t) \frac{\partial}{\partial t}f(\vec{p};t) 
\end{align*}
考虑等式右侧第一项则有:
\[
	\int \,d^3\vec{p} \frac{\partial}{\partial t} f(\vec{p};t) = \frac{d}{dt} \int \,d^3\vec{p} f(\vec{p};t) = \frac{d\rho}{dt} = 0.
\]
其中$\rho$为实空间中粒子数密度,由于均匀且无外力作用,则有:$\frac{d\rho}{dt} = 0$。将Boltzmann方程带入上式可得:
\[
	\frac{dH}{dt} = \int \,d^3\vec{p} \,d^3\vec{p}_2 \,d^3\vec{p}_1' \,d^3\vec{p}_2' \vert T_{fi} \vert^2 \delta^4(P_f-P_i) [f(\vec{p}_1';t)f(\vec{p}_2';t) - f(\vec{p};t)f(\vec{p}_2;t)] \ln f(\vec{p};t).
\]
交换$\vec{p}$和$\vec{p}_2$以及$\vec{p}_1'$和$\vec{p}_2'$可得:
\[
	\frac{dH}{dt} = \int \,d^3\vec{p} \,d^3\vec{p}_2 \,d^3\vec{p}_1' \,d^3\vec{p}_2' \vert T_{fi} \vert^2 \delta^4(P_f-P_i) [f(\vec{p}_1';t)f(\vec{p}_2';t) - f(\vec{p};t)f(\vec{p}_2;t)] \ln f(\vec{p}_2;t).
\]
对上面两式利用时间反演不变性可得:
\[
	\frac{dH}{dt} = \int \,d^3\vec{p} \,d^3\vec{p}_2 \,d^3\vec{p}_1' \,d^3\vec{p}_2' \vert T_{fi} \vert^2 \delta^4(P_f-P_i) [f(\vec{p};t)f(\vec{p}_2;t) - f(\vec{p}_1';t)f(\vec{p}_2';t)] \ln f(\vec{p}_1';t).
\]
\[
	\frac{dH}{dt} = \int \,d^3\vec{p} \,d^3\vec{p}_2 \,d^3\vec{p}_1' \,d^3\vec{p}_2' \vert T_{fi} \vert^2 \delta^4(P_f-P_i) [f(\vec{p};t)f(\vec{p}_2;t) - f(\vec{p}_1';t)f(\vec{p}_2';t)] \ln f(\vec{p}_2';t).
\]
则有:
\begin{align*}
	\frac{dH}{dt} =& \frac{1}{4} \int \,d^3\vec{p} \,d^3\vec{p}_2 \,d^3\vec{p}_1' \,d^3\vec{p}_2' \vert T_{fi} \vert^2 \delta^4(P_f-P_i) \\
	&\times[f(\vec{p}_1';t)f(\vec{p}_2';t) - f(\vec{p};t)f(\vec{p}_2;t)] \ln \frac{f(\vec{p};t)f(\vec{p}_2;t)}{f(\vec{p}_1';t)f(\vec{p}_2';t)}.
\end{align*}
注意到如果$f(\vec{p}_1';t)f(\vec{p}_2';t) - f(\vec{p};t)f(\vec{p}_2;t)>0$,则有:$\ln \frac{f(\vec{p};t)f(\vec{p}_2;t)}{f(\vec{p}_1';t)f(\vec{p}_2';t)}<0$; \\
如果$f(\vec{p}_1';t)f(\vec{p}_2';t) - f(\vec{p};t)f(\vec{p}_2;t)<0$,则有:$\ln \frac{f(\vec{p};t)f(\vec{p}_2;t)}{f(\vec{p}_1';t)f(\vec{p}_2';t)}>0$; \\
如果$f(\vec{p}_1';t)f(\vec{p}_2';t) - f(\vec{p};t)f(\vec{p}_2;t)=0$,则有:$\ln \frac{f(\vec{p};t)f(\vec{p}_2;t)}{f(\vec{p}_1';t)f(\vec{p}_2';t)}=0$; \\
综上所述,有
\[
	\frac{dH}{dt} \leq 0.
\]
\end{tcolorbox}
\end{enumerate}

\end{document}