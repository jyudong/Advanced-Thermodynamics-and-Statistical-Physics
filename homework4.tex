\documentclass[reqno,a4paper,12pt]{amsart}

\usepackage{amsmath,amssymb,amsthm,geometry,xcolor,soul,graphicx}
\usepackage{titlesec}
\usepackage{enumerate}
\usepackage{lipsum}
\usepackage{listings}
\RequirePackage[most]{tcolorbox}
\usepackage{xeCJK}
\setCJKmainfont{Kai}

\geometry{left=0.7in, right=0.7in, top=1in, bottom=1in}



\renewcommand{\baselinestretch}{1.3}

\title{高等热力学与统计物理第四次作业}
\author{董建宇}

\begin{document}
\maketitle

\titleformat{\section}[hang]{\small}{\thesection}{0.8em}{}{}
\titleformat{\subsection}[hang]{\small}{\thesubsection}{0.8em}{}{}

\section{计算下列系统内电子气体的Fermi能,Fermi温度和Fermi速度}
\begin{enumerate}
	\item 室温下的金属钠:密度为$0.97g/cm^3$,每个原子贡献一个传导电子,假设他们的能量和动量关系为$E = \frac{p^2}{2m}$,即忽略晶格场对电子运动的影响。
	\begin{tcolorbox}[breakable, colback = black!5!white, colframe = black]
	在0K下,费米狄拉克分布为:
	\[
		f = \left\{ \begin{aligned}
			&1, & E<\epsilon_F, \\
			&0, & E>\epsilon_F.
		\end{aligned}\right.
	\]
	则粒子数N为:
	\begin{align*}
		N &= 2\left( \frac{L}{2\pi} \right)^3 \iiint \,dk_x\,dk_y\,dk_z \\
		&= 2 \frac{V}{(2\pi)^3} \int 4\pi k^2\,dk \\
		&= \frac{V}{2\pi^2} \frac{(2m)^{3/2}}{\hbar^3} \int_0^{\epsilon_F} \sqrt{E}\,dE \\
		&= \frac{V}{3\pi^2} \frac{(2m)^{3/2}}{\hbar^3}\epsilon_F^{3/2}.
	\end{align*}
	其中粒子数$N = nN_A$,体积$V = \frac{nM}{\rho}$,$n$为物质的量,$N_A$为阿伏伽德罗常数,$M$为钠的摩尔质量。则有:
	\[
		\epsilon_F = \frac{1}{2m} \left( \frac{3\pi^2\hbar^3 N_A\rho}{M} \right)^{2/3}.
	\]
	代入数据可得费米能量为:
	\[
		\epsilon_F = 5.048\times 10^{-19} J = 3.15 eV.
	\]
	费米温度为:
	\[
		T_F = \frac{\epsilon_F}{k_B} = 36562K.
	\]
	费米速度为:
	\[
		v_F = \sqrt{\frac{2\epsilon_F}{m}} = 1.05\times 10^{6}m/s.
	\]
	\end{tcolorbox}
	
	\item 天狼星的伴星(白矮星):其质量约为太阳质量的0.98倍,半径约为太阳半径的0.0084倍。假设星体全部由氦构成。
	\begin{tcolorbox}[breakable, colback = black!5!white, colframe = black]
	对于白矮星而言,密度为:
	\[
		\rho = \frac{0.98m_{sun}}{\frac{4}{3}\pi (0.0084r_{sun})^3} = 2.315\times 10^{9} kg/m^3.
	\]
	每一个氦原子贡献2个电子,考虑相对论条件粒子数满足:
	\[
		N = \frac{2V}{\pi^2} \int_0^{k_F} k^2\,dk = \frac{2V}{3\pi^2}k_F^3.
	\]
	则有:
	\[
		k_F = \sqrt[3]{\frac{6\pi^2N_A\rho}{M}} = 2.743\times 10^{12} m^{-1}.
	\]
	费米能量为:
	\[
		\epsilon_F = \sqrt{p^2c^2+m^2c^4} - mc^2 = 3.739\times 10^{-14}J = 2.33 \times 10^5 eV.
	\]
	费米温度为:
	\[
		T_F = \frac{\epsilon_F}{k_B} = 2.708\times 10^{9}K.
	\]
	费米速度为:
	\[
		v_F = \sqrt{\frac{p^2c^2}{p^2+m^2c^2}} = 2.180\times 10^{8}m/s.
	\]
	\end{tcolorbox}
\end{enumerate}

\section{证明非相对论简并电子气体的热力学函数是
\begin{align*}
	G &= N\mu = N\epsilon_F\left[ 1-\frac{1}{12}\pi^2\left( \frac{kT}{\epsilon_F} \right)^2 - \frac{1}{80}\pi^4\left( \frac{kT}{\epsilon_F} \right)^4 \dots \right] \\
	E &= \frac{3}{5}N\epsilon_F \left[ 1 + \frac{5}{12}\pi^2\left( \frac{kT}{\epsilon_F} \right)^2 - \frac{1}{16}\pi^4\left( \frac{kT}{\epsilon_F} \right)^4 \dots \right] \\
	C_V &= \frac{1}{2}N\pi^2 \frac{k^2T}{\epsilon_F} \left[ 1- \frac{3}{10}\pi^2\left( \frac{kT}{\epsilon_F} \right)^2 \dots \right] \\
	S &= \frac{1}{2}N\pi^2 \frac{k^2T}{\epsilon_F} \left[ 1-\frac{1}{10} \pi^2 \left( \frac{kT}{\epsilon_F} \right)^2 \dots \right]
\end{align*}
其中$N = \frac{V}{3\pi^2} \left( \frac{2m\varepsilon_F}{\hbar} \right)^{3/2}$为电子总数,$\dots$代表$\frac{kT}{\varepsilon_F}$的更高阶项。
}
\begin{tcolorbox}[breakable, colback = black!5!white, colframe = black]
粒子数$N$满足:
\begin{align*}
	N &= \frac{V(2m)^{3/2}}{2\pi^2\hbar^3} \int_0^{\infty} \frac{\sqrt{\epsilon}}{e^{\beta(\epsilon-\mu)}+1} \,d\epsilon \\
	&= \frac{V(2m)^{3/2}}{2\pi^2\hbar^3} \int_0^{\infty} \frac{2\beta}{3} \epsilon^{3/2} \frac{e^{\beta(\epsilon-\mu)}}{\left( e^{\beta(\epsilon-\mu)} + 1 \right)^2} \,d\epsilon.
\end{align*}
内能为:
\begin{align*}
	E &= \frac{V(2m)^{3/2}}{2\pi^2\hbar^3}\int_0^{\infty} \frac{\epsilon^{3/2}}{e^{\beta(\epsilon - \mu)} + 1}\,d\epsilon \\ 
	&= \frac{V(2m)^{3/2}}{2\pi^2\hbar^3} \int_0^{\infty} \frac{2\beta}{5}\epsilon^{5/2} \frac{e^{\beta(\epsilon-\mu)}}{\left( e^{\beta(\epsilon-\mu)} + 1 \right)^2} \,d\epsilon.
\end{align*}
令$\xi = \beta(\epsilon - \mu)$,则积分化为:
\[
	N = \frac{V(2m)^{3/2}}{3\pi^2\hbar^3} \int_{-\beta\mu}^{\infty} (kT\xi+\mu)^{3/2} \frac{e^{\xi}}{(e^{\xi} + 1)^2}\,d\xi.
\]
\[
	E = \frac{V(2m)^{3/2}}{5\pi^2\hbar^3} \int_{-\beta\mu}^{\infty} (kT\xi + \mu)^{5/2} \frac{e^{\xi}}{(e^{\xi} + 1)^2}\,d\xi.
\]
令
\[
	I = \int_{-\beta\mu}^{\infty} (kT\xi+\mu)^{3/2} \frac{e^{\xi}}{(e^{\xi} + 1)^2}\,d\xi \approx \int_{-\infty}^{\infty} (kT\xi+\mu)^{3/2} \frac{e^{\xi}}{(e^{\xi} + 1)^2}\,d\xi.
\]
对$(kT\xi+\mu)^{3/2}$做泰勒展开得:
\[
	(kT\xi+\mu)^{3/2} = \mu^{3/2} + \frac{3}{2}kT\mu^{1/2}\xi + \frac{3}{8}(kT)^2\mu^{-1/2}\xi^2 - \frac{1}{16}(kT)^3\mu^{-3/2}\xi^3 + \frac{3}{128}(kT)^4\mu^{-5/2}\xi^4 + \mathcal{O}(\xi^4).
\]
对$(kT\xi+\mu)^{5/2}$做泰勒展开可得:
\[
	(kT\xi+\mu)^{5/2} = \mu^{5/2} + \frac{5}{2}kT\mu^{3/2}\xi + \frac{15}{8}(kT)^2\mu^{1/2}\xi^2 + \frac{5}{16}(kT)^3\mu^{-1/2}\xi^3 - \frac{5}{128}(kT)^4\mu^{-3/2}\xi^4+\mathcal{O}(\xi^4).
\]
由于$\frac{e^\xi}{(e^\xi + 1)^2}$是关于$\xi$的偶函数,则泰勒展开只有偶次幂的积分不为零。令$c_n = \int_0^{\infty} \frac{\xi^{2n}e^{\xi}}{(e^{\xi}+1)^2}\,d\xi$。当$n = 0$时,$c_0 = \frac{1}{2}$。当$n>0$时,有:
\[
	\frac{e^\xi}{(e^\xi + 1)^2} = \frac{e^{-\xi}}{(e^{-\xi} + 1)^2} = \sum_{l=1}^{\infty} (-1)^{l-1} l e^{-l\xi}.
\]
则有:
\[
	c_n = \sum_{l=1}^{\infty} (-1)^{l-1} l \int_0^{\infty} \xi^{2n}e^{-l\xi} \,d\xi = (2n)! \sum_{l=1}^{\infty} \frac{(-1)^{l-1}}{l^{2n}}.
\]
则有:
\[
	c_1 = 2\left( \sum_{l = odd} \frac{1}{l^2} - \sum_{l=even} \frac{1}{l^2} \right) = 2\left( \sum_{l=1}^{\infty} \frac{1}{l^2} - \frac{2}{2^2} \sum_{l=1}^{\infty} \frac{1}{l^2} \right) = \frac{\pi^2}{6}.
\]
\[
	c_2 = 24 \left( \sum_{l=odd}\frac{1}{l^4} - \sum_{l=even} \frac{1}{l^4} \right) = 24 \left( \sum_{l=1}^{\infty} \frac{1}{l^4} - \frac{2}{16}\sum_{l=1}^{\infty} \frac{1}{l^4} \right) = \frac{7\pi^4}{30}.
\]
则有:
\[
	N = \frac{V(2m)^{3/2}}{3\pi^2\hbar^3} \mu^{3/2} \left[ 1 + \frac{\pi^2}{8}\left( \frac{kT}{\mu} \right)^2 + \frac{7\pi^4}{640} \left( \frac{kT}{\mu} \right)^4 + \dots \right].
\]
\[
	E = \frac{V(2m)^{3/2}}{5\pi^2\hbar^3}\mu^{5/2}\left[ 1 + \frac{5\pi^2}{8}\left( \frac{kT}{\mu} \right)^2 - \frac{7\pi^4}{384}\left( \frac{kT}{\mu} \right)^4 \cdots \right].
\]
则有:
\[
	E = \frac{3}{5}N\mu \left[ 1 + \frac{\pi^2}{2}\left( \frac{kT}{\mu} \right)^2 - \frac{11\pi^4}{120}\left( \frac{kT}{\mu} \right)^4 \right].
\]
对比
\[
	N = \frac{V(2m)^{3/2}}{3\pi^2\hbar^3}\epsilon_F^{3/2}.
\]
有:
\[
	\epsilon_F = \mu \left[ 1 + \frac{\pi^2}{8}\left( \frac{kT}{\mu} \right)^2 + \frac{7\pi^4}{640} \left( \frac{kT}{\mu} \right)^4 + \dots \right]^{2/3}
\]
泰勒展开可得:
\[
	\epsilon_F = \mu\left[ 1 + \frac{\pi^2}{12}\left( \frac{kT}{\mu} \right)^2 + \frac{\pi^4}{180}\left( \frac{kT}{\mu} \right)^4 \right]
\]
利用级数反演可得:
\[
	\mu = \epsilon_F \left[ 1 - \frac{\pi^2}{12}\left( \frac{kT}{\epsilon_F} \right)^2 - \frac{\pi^4}{80}\left( \frac{kT}{\epsilon_F} \right)^4 \cdots \right]
\]
则Gibbs势为:
\[
	G = N\mu = N\epsilon_F \left[ 1 - \frac{\pi^2}{12}\left( \frac{kT}{\epsilon_F} \right)^2 - \frac{\pi^4}{80}\left( \frac{kT}{\epsilon_F} \right)^4 \cdots  \right].
\]
内能为:
\[
	E = \frac{3}{5}N\epsilon_F \left[ 1 + \frac{5\pi^2}{12}\left( \frac{kT}{\epsilon_F} \right)^2 - \frac{\pi^4}{16}\left( \frac{kT}{\epsilon_F} \right)^4 \cdots \right]
\]
则等容热熔为:
\[
	C_V = \left( \frac{\partial E}{\partial T} \right)_V = \frac{1}{2}N\pi^2\frac{k^2T}{\epsilon_F} \left[ 1 - \frac{3}{10}\pi^2\left( \frac{kT}{\epsilon_F} \right)^2 + \cdots \right]
\]
在非相对论条件下有:
\[
	PV = \frac{2}{3}E.
\]
则气体自由能为:
\[
	F = G - PV = G - \frac{2}{3}E = N\epsilon_F\left[ \frac{3}{5} - \frac{\pi^2}{4}\left( \frac{kT}{\epsilon_F} \right)^2 + \frac{\pi^4}{80}\left( \frac{kT}{\epsilon_F} \right)^4 + \dots \right]
\]
则熵为:
\[
	S = -\left( \frac{\partial F}{\partial T} \right)_V = N\epsilon_F\left( \frac{\pi^2}{2}\frac{k^2T}{\epsilon_F^2} - \frac{\pi^4}{20}\frac{k^4T^3}{\epsilon_F^4} + \cdots \right).
\]

\end{tcolorbox}


\end{document}